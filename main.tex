\documentclass{article}
\usepackage[utf8]{inputenc}
\usepackage[T1]{fontenc}
\usepackage[polish]{babel}
\usepackage{graphicx}
\usepackage[margin=1in]{geometry}

\title{Refleks}
\date{12 Czerwca 2025}

\begin{document}

\maketitle

\vspace{-2ex}

\begin{center}
\begin{tabular}{|p{4cm}|p{5cm}|p{4cm}|}
\hline
\textbf{Imię i Nazwisko} & \textbf{Email} & \textbf{Udział w projekcie} \\
\hline
Igor Ordecha (Lider) & 251601@edu.p.lodz.pl & Udział: 33\% \\
Adrian Urbańczyk & 252960@edu.p.lodz.pl & Udział: 33\% \\
Jakub Bukowski & 251492@edu.p.lodz.pl & Udział: 33\% \\
\hline
\end{tabular}
\end{center}

\vspace{2ex}

\noindent\textbf{Aktualnie wykorzystywany sprzęt:} \\
Na potrzeby projektu wykorzystano własny sprzęt: \\
Raspberry Pi Pico W - RP2040 ARM Cortex M0+ CYW43439 - WiFi.

\vspace{2ex}

\noindent\textbf{Zakres projektu:}

\begin{center}
\begin{tabular}{|c|p{5cm}|c|c|}
\hline
\textbf{Lp.} & \textbf{Założona funkcjonalność} & \textbf{Stan} & \textbf{Osoba odpowiedzialna} \\
\hline
1 & GPIO & Działa & Igor Ordecha \\
2 & I2C & Działa & Adrian Urbańczyk \\
3 & Timer & Działa & --- \\
4 & SPI & Działa & Jakub Bukowksi \\
5 & LCD & Działa & Adrian Urbańczyk \\
6 & RFID & Działa & Jakub Bukowksi \\
\hline
\end{tabular}
\end{center}

\vspace{2ex}


\section{Wprowadzenie}

\end{document}
